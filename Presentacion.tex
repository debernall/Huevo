\documentclass[11pt]{beamer}
\usepackage[utf8]{inputenc}
\usepackage[T1]{fontenc}
\usepackage{lmodern}
\usepackage[spanish]{babel}
\usetheme{metropolis}


\begin{document}
	\author{Daniel Eduardo Bernal Lozano}
	\title{Proyecto Huevo}
	\subtitle{Elementos finitos}
	%\logo{}
	%\institute{}
	\date{19 de Noviembre, 2023}
	%\subject{}
	%\setbeamercovered{transparent}
	%\setbeamertemplate{navigation symbols}{}
	\begin{frame}[plain]
		\maketitle
	\end{frame}
	
	
	\begin{frame}
		\frametitle{Proyecto Huevo}
		
		Simulación de la evolución de la temperatura interna de un huevo en proceso de incubación durante las primeras horas de desarrollo.
		
			\begin{figure}
			\centering
			\includegraphics[width=0.9\linewidth]{DistTemp.jpg}
			\caption{Temperatura del huevo durante el tiempo inicial.}
		\end{figure}
		
		
		
	\end{frame}

	\begin{frame}
	\frametitle{Proyecto Huevo}
	
	Forma del huevo y sus estructuras:
		\begin{equation}
		x^2/a^2+y^2/(b+xtan\theta)=1 
	\end{equation}

	\begin{equation}
	x^2 + y^2 + z^2 = R^2
\end{equation}

	Crecimiento de la yema del huevo:
	
	\begin{equation}
		\frac{dV}{dt}= const * V (log V_{max}-log V)
	\end{equation}
	
\end{frame}



\begin{frame}
	\frametitle{Proyecto Huevo}
	
	Ecuación de difusión de calor:
	\begin{equation}
		\frac{\partial T}{\partial t} = \frac{k}{\rho c_p} \nabla^2 T
	\end{equation}
	
	Diferencias finitas:

	
	\begin{equation*}
		T^{n+1}_{i,j,k} = Fo(T^{n}_{i,j,k}(1/Fo -6)+T^{n}_{i+1,j,k}+T^{n}_{i-1,j,k}+T^{n}_{i,j+1,k}+T^{n}_{i,j-1,k}+
	
	\end{equation*}

	\begin{equation*}
		T^{n}_{i,j,k+1}+T^{n}_{i,j,k-1} + Q  \Delta x/k) 
	\end{equation*}

		\begin{equation}
			Fo = \frac{k\Delta t}{\rho c_p (\Delta x)^2}
	\end{equation}
	
	
\end{frame}
	
	


	\begin{frame}
	\frametitle{Proyecto Huevo}
	
	Para contrastar los resultados obtenidos simulé la cocción de un huevo.
	
		\begin{figure}
		\centering
		\includegraphics[width=0.6\linewidth]{TemperaturaCopia.jpg}
		\caption{Temperatura del huevo en cocción para diferentes periodos de tiempo.}
	\end{figure}
	
	
	
\end{frame}



	\begin{frame}
		\frametitle{Proyecto Huevo}
		Modelar un sistema que en el cual interactúan diferentes variables. Volumen, temperatura.
		
	\begin{figure}
	\centering
	\includegraphics[width=0.65\linewidth]{TemperaturaMedia.jpg}
	\caption{Temperatura media de las diferentes estructuras del modelo durante 3 horas.}
\end{figure}
	
	\end{frame}

	\begin{frame}
		\frametitle{Como lo hice}
		C++ -> evolución temporal\\
		Python -> Graficar\\
		Latex -> Documento\\
		Make -> Compilar documentos\\
	
				\begin{figure}
			\centering
			\includegraphics[width=0.9\linewidth]{DistTemp.jpg}
			\caption{Temperatura del huevo durante el tiempo inicial.}
		\end{figure}
		
		
		

		
		
	\end{frame}

	\begin{frame}
		\frametitle{Resultados}
			\begin{figure}
			\centering
			\includegraphics[width=0.8\linewidth]{TemperaturaMedia.jpg}
			\caption{Temperatura media de las diferentes estructuras del modelo durante 3 horas.}
		\end{figure}
	\end{frame}

	\begin{frame}
		\frametitle{Resultados}
		\begin{figure}
			\centering
			\includegraphics[width=0.7\linewidth]{Volumen.jpg}
			\caption{Volumen de las diferentes estructuras del modelo durante 3 horas.}
		\end{figure}
	\end{frame}


	\begin{frame}
	\frametitle{Resultados}
		\begin{figure}
		\centering
		\includegraphics[width=0.7\linewidth]{DistTempFinal.jpg}
		\caption{Temperatura del huevo durante el tiempo inicial.}
	\end{figure}
\end{frame}

\begin{frame}
	\frametitle{Resultados}
		\begin{figure}[h]
		\centering
		\includegraphics[width=0.7\linewidth]{TemperaturaMediaCopia.jpg}
		\caption{Evolución temporal de la temperatura media del huevo en cocción.}
	\end{figure}
\end{frame}


	\begin{frame}
		\frametitle{Conclusiones}
		La comparación con los resultados de la cocción del huevo muestran que el método aplicado es una buena aproximación al modelo.\\
		
		El método falla al adicionar términos de convección y generación de calor.\\
		
		La definición de la red de nodos no es la adecuada ya que aparecen nodos aislados dentro de estructuras bien definidas.\\
		
		La representación de los intervalos de tiempo es muy pequeña, por tanto no es posible obtener resultados fiables al evolucionar el sistema durante largos periodos de tiempo.\\
		
		Diferencias finitas es un buen método para aprender a resolver sistemas de ecuaciones.\\
		
	\end{frame}

	\begin{frame}
		\frametitle{Siguientes pasos}
		Definir de manera adecuada la red de nodos.\\
		Implementar métodos mas robustos como elementos finitos.\\
		Incluir términos de convección, radiación y generación.\\
		Incluir variables que influyen en la generación de calor como el transporte de oxigeno a través de la cáscara hasta el embrión.
	\end{frame}

	\begin{frame}
		\frametitle{Referencias}
		1. D.C. Deeming y M.W.J. Ferguson. (1992) Egg incubation: its effects on embryonic development in birds and reptiles.\\
		
		
		2. B.M. Freeman. (1974) Development of the avian embryo.\\
		
		3. D.R. Croft, David G. Lilley (1997) Heat Transfer Calculations Using Finite Difference Equations\\
		
		4. P Roura, J Fort and J Saurina (2000) How long does it take to boil an egg?. A simple approach to the energy transfer equation.\\
	\end{frame}


\end{document}